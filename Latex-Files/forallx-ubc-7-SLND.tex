%!TEX root = forallx-ubc.tex
\chapter{Natural Deduction Proofs in SL}
\label{ch.ND.proofs}

This chapter introduces a different proof system in SL, separate from the tree method. The tree method has advantages and disadvantages. One advantage of trees is that, for the most part, they can be produced in a purely mechanical way; another is that, when a tree remains open, the tree method gives us a recipe for constructing an interpretation that satisfies the root. One disadvantage is that they do not always emphasize in an intuitive way why a conclusion follows from a set of premises; they show that something \emph{must} be the case, on pain of contradiction, but they don't always demonstrate, in a way closely connected to natural reasoning, why some things follow from other things.

The \define{natural deduction} system of this chapter will differ from the tree method in all of these respects. It is intended to model human reasoning in a closer way, illustrating the connections between various claims; consequently, working through a natural deduction proof requires a bit more insight and inspiration than a tree proof does.

Natural deduction proofs can be used to prove that an argument is valid; if an argument is invalid, our natural deduction system will not necessarily make that obvious. (Any valid argument in SL can be shown to be valid via a natural deduction proof--- i.e., this method is complete too (and sound)--- but that's no guarantee that any individual will be able to come up with the proof.)

Consider two arguments in SL:

\begin{multicols}{2}
Argument A
\begin{earg}
\item[] $P \eor Q$
\item[] $\enot P$
\item[\therefore] Q
\end{earg}

Argument B
\begin{earg}
\item[] $P \eif Q$
\item[] $P$
\item[\therefore] Q
\end{earg}

\end{multicols}

Clearly, these are valid arguments. You can confirm that they are valid by constructing four-line truth tables, or by using trees. Argument A makes use of an inference form that is always valid: Given a disjunction and the negation of one of the disjuncts, the other disjunct follows as a valid consequence. This rule is called \emph{disjunctive syllogism}.

Argument B makes use of a different valid form: Given a conditional and its antecedent, the consequent follows as a valid consequence. This is called \emph{modus ponens}.

When we construct truth tables, we do not need to give names to different inference forms. There is no reason to distinguish modus ponens from a disjunctive syllogism. For this same reason, however, the method of truth tables does not clearly show \emph{why} an argument is valid. If you were to do a 1024-line truth table for an argument that contains ten sentence letters, then you could check to see if there were any lines on which the premises were all true and the conclusion were false. If you did not see such a line and provided you made no mistakes in constructing the table, then you would know that the argument was valid. Yet you would not be able to say anything further about why this particular argument was a valid argument form.

We aim to show that particular arguments are valid in a way that allows us to understand the reasoning involved in the argument. We begin with basic argument forms, like disjunctive syllogism and modus ponens. These forms can then be combined to make more complicated arguments, like this one:
\begin{earg}
\item[(1)] $\enot L \eif (J \eor L)$
\item[(2)] $\enot L$
\item[\therefore] $J$
\end{earg}
By modus ponens, (1) and (2) entail $J \eor L$. This is an \emph{intermediate conclusion}. It follows logically from the premises, but it is not the conclusion we want. Now $J \eor L$ and (2) entail $J$, by disjunctive syllogism. We do not need a new rule for this argument. The proof of the argument shows that it is really just a combination of rules we have already introduced.

Formally, a \define{proof} is a sequence of sentences. The first sentences of the sequence are assumptions; these are the premises of the argument. Every sentence later in the sequence follows from earlier sentences by one of the rules of proof. The final sentence of the sequence is the conclusion of the argument.

\section{Basic rules for SL}

In designing a proof system, we could just start with disjunctive syllogism and modus ponens. Whenever we discovered a valid argument which could not be proven with rules we already had, we could introduce new rules. Proceeding in this way, we would have an unsystematic grab bag of rules. We might accidentally add some strange rules, and we would surely end up with more rules than we need.

Instead, we will develop what is called a \define{natural deduction} system. In a natural deduction system, there will be two rules for each logical operator: an \define{introduction} rule that allows us to prove a sentence that has it as the main logical operator and an \define{elimination} rule that allows us to prove something given a sentence that has it as the main logical operator.

In addition to the rules for each logical operator, we will also have a reiteration rule. If you already have shown something in the course of a proof, the reiteration rule allows you to repeat it on a new line. For instance:

\begin{proof}
	\have{a1}\metaA{}
	\have{a2}\metaA{} \by{R}{a1}
\end{proof}

The numbers on the left indicate line numbers; they are used for reference in justifying later steps in the proof. New lines in a proof must always be justified by rules and reference to previous lines. The `R 1' to the right on line 2 is the justification for that line--- that line is permitted by the reiteration rule (R), applied to line 1.

Obviously, the reiteration rule will not allow us to show anything \emph{new}. For that, we will need more rules. The remainder of this section will give introduction and elimination rules for all of the sentential connectives. This will give us a complete proof system for SL.

All of the rules introduced in this chapter are summarized in the Quick Reference guide at the end of this book.

\subsection{Conjunction}

What would you need to show in order to prove $E \eand F$? The natural answer is, you'd need to prove $E$, and you'd also need to prove $F$. In fact this holds much more generally; one can prove any conjunction \metaA{}\eand\metaB{} by proving \metaA{} and also proving \metaB{}, whether or not these conjuncts are atomic sentences. If you can prove $[(A \eor J) \eif V]$ and  $[(V \eif L) \eiff (F \eor N)]$, then you have effectively proven
$$[(A \eor J) \eif V] \eand [(V \eif L) \eiff (F \eor N)].$$
So this will be our conjunction introduction rule, which we abbreviate `{\eand}I':

\begin{proof}
	\have[m]{a}\metaA{}
	\have[n]{b}\metaB{}
	\have[\ ]{c}{\metaA{}\eand\metaB{}} \ai{a, b}
\end{proof}

As always, the \metaA{} and \metaB{} stand in for any arbitrary sentence in SL; the $m$ and $n$ here stand in for arbitrary line numbers. In an actual proof, the lines are numbered $1, 2, 3, \ldots$ and rules must be applied to specific line numbers. When we define the rule, however, we use variables to underscore the point that the rule may be applied to any two lines that are already in the proof. Note that it is not a requirement that $m$ and $n$ be consecutive lines, or that they appear in the order listed here. We require only that each line has been established somewhere above in the proof. If you have $K$ on line 15 and $L$ on line 8, you can prove $(K\eand L)$ at some later point in the proof with the justification `{\eand}I 15, 8.' (By convention, we won't always worry about which order to write the line numbers down in. `{\eand}I 8, 15' is OK too.)

Now, consider the elimination rule for conjunction. What are you entitled to conclude from a sentence like $E \eand F$? Surely, you are entitled to conclude $E$; if $E \eand F$ were true, then $E$ would be true. Similarly, you are entitled to conclude $F$. This will be our conjunction elimination rule, which we abbreviate `{\eand}E':

\begin{proof}
	\have[m]{ab}{\metaA{}\eand\metaB{}}
	\have[\ ]{a}\metaA{} \ae{ab}
	\have[\ ]{b}\metaB{} \ae{ab}
\end{proof}

When you have a conjunction on some line of a proof, you can use {\eand}E to derive either of the conjuncts. This rule allows either of the two developments listed here. You may also apply the rule twice, to get both. The {\eand}E rule requires only one sentence, so we write one line number as the justification for applying it.

Even with just these two rules, we can provide some proofs. Consider this argument.
\begin{earg}
\item[] $[(A\eor B)\eif(C\eor D)] \eand [(E \eor F) \eif (G\eor H)]$
\item[\therefore] $[(E \eor F) \eif (G\eor H)] \eand [(A\eor B)\eif(C\eor D)]$
\end{earg}
The main logical operator in both the premise and conclusion is conjunction. Since conjunction is symmetric, the argument is obviously valid. In order to provide a proof, we begin by writing down the premise. After the premises, we draw a horizontal line--- everything below this line must be justified by a rule of proof. So the beginning of the proof looks like this:

\begin{proof}
	\hypo{ab}{{[}(A\eor B)\eif(C\eor D){]} \eand {[}(E \eor F) \eif (G\eor H){]}}
\end{proof}

From the premise, we can get each of the conjuncts by {\eand}E. The proof now looks like this:

\begin{proof}
	\hypo{ab}{{[}(A\eor B)\eif(C\eor D){]} \eand {[}(E \eor F) \eif (G\eor H){]}}
	\have{a}{{[}(A\eor B)\eif(C\eor D){]}} \ae{ab}
	\have{b}{{[}(E \eor F) \eif (G\eor H){]}} \ae{ab}
\end{proof}

The rule {\eand}I requires that we have each of the conjuncts available somewhere in the proof. They can be separated from one another, and they can appear in any order. So by applying the {\eand}I rule to lines 3 and 2, we arrive at the desired conclusion. The finished proof looks like this:

\begin{proof}
	\hypo{ab}{{[}(A\eor B)\eif(C\eor D){]} \eand {[}(E \eor F) \eif (G\eor H){]}}

	\have{a}{{[}(A\eor B)\eif(C\eor D){]}} \ae{ab}
	\have{b}{{[}(E \eor F) \eif (G\eor H){]}} \ae{ab}
	\have{ba}{{[}(E \eor F) \eif (G\eor H){]} \eand {[}(A\eor B)\eif(C\eor D){]}} \ai{a,b}
\end{proof}

This proof is not terribly interesting, but it shows how we can use rules of proof together to demonstrate the validity of an argument form. Note also that using a truth table to show that this argument is valid would have required a staggering 256 lines, since there are eight sentence letters in the argument. A proof via trees would be less unwieldy than that, but it would be less simple and elegant than this one. (Constructing such a proof would be a good exercise for tree review.)


\subsection{Disjunction}
If $M$ is true, then $M \eor N$ must also be true. In general, the disjunction introduction rule ({\eor}I) allows us to derive a disjunction if we have one of the two disjuncts:

\begin{proof}
	\have[m]{a}\metaA{}
	\have[\ ]{ab}{\metaA{}\eor\metaB{}}\oi{a}
	\have[\ ]{ba}{\metaB{}\eor\metaA{}}\oi{a}
\end{proof}

Notice that \metaB{} can be \emph{any} sentence whatsoever. So the following is a legitimate proof:

\begin{proof}
	\hypo{m}{M}
	\have{mmm}{M \eor ([(A\eiff B) \eif (C \eand D)] \eiff [E \eand F])}\oi{m}
\end{proof}

It may seem odd that just by knowing $M$ we can derive a conclusion that includes sentences like $A$, $B$, and the rest--- sentences that have nothing to do with $M$. Yet the conclusion follows immediately by {\eor}I. This is as it should be: The truth conditions for the disjunction mean that, if \metaA{} is true, then $\metaA{}\eor \metaB{}$ is true regardless of what \metaB{} is. So the conclusion could not be false if the premise were true; the argument is valid.

Now consider the disjunction elimination rule. What can you conclude from $M \eor N$? You cannot conclude $M$. It might be $M$'s truth that makes $M \eor N$ true, as in the example above, but it might not. From $M \eor N$ alone, you cannot conclude anything about either $M$ or $N$ specifically. If you also knew that $N$ was false, however, then you would be able to conclude $M$.

This is the disjunctive syllogism rule mentioned in the introduction. It will be our official disjunction elimination rule ({\eor}E). If you have a disjunction and also the negation of one of its disjuncts, you may conclude the other disjunct.

\begin{multicols}{2}
\begin{proof}
	\have[m]{ab}{\metaA{}\eor\metaB{}}
	\have[n]{nb}{\enot\metaB{}}
	\have[\ ]{a}\metaA{} \oe{ab,nb}
\end{proof}

\begin{proof}
	\have[m]{ab}{\metaA{}\eor\metaB{}}
	\have[n]{na}{\enot\metaA{}}
	\have[\ ]{b}\metaB{} \oe{ab,nb}
\end{proof}

\end{multicols}

We represent two different inference patterns here, because the rule allows you to conclude \emph{either} disjunct from the negation of the other. If we'd only listed the left version of the rule above, then {\eor}E would've only permited one to conclude the \emph{first} disjunct from the negation of the \emph{second} one, along with the disjunction. Our rule lets us work with either disjunction. (If you want to be very fussy about it, you could think of these as two different rules with a strong conceptual similarity that happen to have the same name.)

\subsection{Conditional}

Consider this argument:
\begin{earg}
\item[] $R \eor F$
\item[\therefore] $\enot R \eif F$
\end{earg}
The argument seems like it should be valid. (You can confirm this by examining the truth tables.) What should the conditional introduction rule be, such that we can draw this conclusion?

We begin the proof by writing down the premise of the argument and drawing a horizontal line, like this:

\begin{proof}
	\hypo{rf}{R \eor F}
\end{proof}

If we had $\enot R$ as a further premise, we could derive $F$ by the {\eor}E rule. We do not have $\enot R$ as a premise of this argument, nor can we derive it directly from the premise we do have--- so we cannot simply prove $F$. What we will do instead is start a \emph{subproof}, a proof within the main proof. When we start a subproof, we draw another vertical line to indicate that we are no longer in the main proof. Then we write in an assumption for the subproof. This can be anything we want. Here, it will be helpful to assume $\enot R$. Our proof now looks like this:

\begin{proof}
	\hypo{rf}{R \eor F}
	\open
		\hypo{nr}{\enot R}
	\close
\end{proof}

It is important to notice that we are not claiming to have proven $\enot R$. We do not need to write in any justification for the assumption line of a subproof. The vertical line indicates that an \emph{assumption} is being made. You can think of the subproof as posing the question: What could we show \emph{if} $\enot R$ were true? For one thing, we can derive $F$. So we do:

\begin{proof}
	\hypo{rf}{R \eor F}
	\open
		\hypo{nr}{\enot R}
		\have{f}{F}\oe{rf, nr}
	\close
\end{proof}

This has shown that \emph{if} we had $\enot R$ as a premise, \emph{then} we could prove $F$. In effect, we have proven $\enot R \eif F$. So the conditional introduction rule ({\eif}I) will allow us to close the subproof and derive $\enot R \eif F$ in the main proof. Our final proof looks like this:

\begin{proof}
	\hypo{rf}{R \eor F}
	\open
		\hypo{nr}{\enot R}
		\have{f}{F}\oe{rf, nr}
	\close
	\have{nrf}{\enot R \eif F}\ci{nr-f}
\end{proof}

The {\eif}I lets us \define{discharge} the assumption we'd been making, ending that vertical line. During lines (2) and (3), we were \emph{assuming} that \enot $R$; by the time we get to line (4), we are no longer making that assumption.

Notice that the justification for applying the {\eif}I rule is the entire subproof. That's why we justify it by reference to a range of lines, instead of a comma-separated list. Usually that will be more than just two lines.

It may seem as if the ability to assume anything at all in a subproof would lead to chaos: Does it allow you to prove any conclusion from any premises? The answer is no, it does not. Consider this proof:

\begin{proof}
	\hypo{a}\metaA{}
	\open
		\hypo{b1}\metaB{}
		\have{b2}\metaB{} \by{R}{b1}
	\close
\end{proof}

Does this show that one can prove any arbitrary sentence \metaB{} from any arbitrary premise \metaA{}? After all, we've written \metaB{} on a line of a proof that began with \metaA{}, without violating any of the rules of our system. The reason this doesn't have that implication is the vertical line that still extends into line 3. That line indicates that the assumption made at line 2 is still in effect. When the vertical line for the subproof ends, the subproof is \emph{closed}. In order to complete a proof, you must close all of the subproofs. The conclusion to be proved must not be `blocked off' by a vertical line; it should be aligned with the premises.

In this example, there is no way to close the subproof and use the R rule again on line 4 to derive \metaB{} in the main proof. Once you close a subproof, you cannot refer back to individual lines inside it. One can only close a subproof via particular rules that allow you to do so; {\eif}I is one such rule. One can't just close a subproof willy-nilly. Closing a subproof is called \emph{discharging} the assumptions of that subproof. So we can put the point this way: You cannot complete a proof until you have discharged all of the assumptions besides the original premises of the argument.

Of course, it is legitimate to do this:

\begin{proof}
	\hypo{a}\metaA{}
	\open
		\hypo{b1}\metaB{}
		\have{b2}\metaB{} \by{R}{b1}
	\close
	\have{bb}{\metaB{}\eif\metaB{}} \ci{b1-b2}
\end{proof}

This should not seem so strange, though. Since \metaB{}\eif\metaB{} is a tautology, no particular premises should be required to validly derive it. (Indeed, as we will see, a tautology follows from any premises.)

Put in its general form, the {\eif}I rule looks like this:

\begin{proof}
	\open
		\hypo[m]{a}\metaA{} \by{(want \metaB{})}{}
		\have[n]{b}\metaB{}
	\close
	\have[\ ]{ab}{\metaA{}\eif\metaB{}}\ci{a-b}
\end{proof}

When we introduce a subproof, we typically write what we want to derive off to the right. This is just so that we do not forget why we started the subproof if it goes on for five or ten lines. There is no `want' rule. It is a note to ourselves, and not formally part of the proof.

Although it is always permissible to open a subproof with any assumption you please, there is some strategy involved in picking a useful assumption. Starting a subproof with an arbitrary, wacky assumption is not a good strategy. It will just waste lines of the proof. In order to derive a conditional by the {\eif}I rule, for instance, you must assume the antecedent of the conditional in a subproof.

The {\eif}I rule also requires that the consequent of the conditional be the last line of the subproof. It is always permissible to close a subproof and discharge its assumptions, but it will not be helpful to do so until you get what you want. This is an illustration of the observation made above, that unlike the tree method, the natural deduction method requires some strategy and thinking ahead.

Now consider the conditional elimination rule. Nothing follows from $M\eif N$ alone, but if we have both $M \eif N$ and $M$, then we can conclude $N$. This rule, modus ponens, will be the conditional elimination rule ({\eif}E).

\begin{proof}
	\have[m]{ab}{\metaA{}\eif\metaB{}}
	\have[n]{a}\metaA{}
	\have[\ ]{b}\metaB{} \ce{ab,a}
\end{proof}

Now that we have rules for the conditional, consider this argument:
\label{proofHS}
\begin{earg}
\item[] $P \eif Q$
\item[] $Q \eif R$
\item[\therefore] $P \eif R$
\end{earg}
We begin the proof by writing the two premises as assumptions. Since the main logical operator in the conclusion is a conditional, we can expect to use the {\eif}I rule. For that, we need a subproof--- so we write in the antecedent of the conditional as assumption of a subproof:

\begin{proof}
	\hypo{pq}{P \eif Q}
	\hypo{qr}{Q \eif R}
	\open
		\hypo{p}{P}
	\close
\end{proof}

We made $P$ available by assuming it in a subproof, allowing us to use {\eif}E on the first premise. This gives us $Q$, which allows us to use {\eif}E on the second premise. Having derived  $R$, we close the subproof. By assuming $P$ we were able to prove $R$, so we apply the {\eif}I rule and finish the proof.

\label{HSproof}
\begin{proof}
	\hypo{pq}{P \eif Q}
	\hypo{qr}{Q \eif R}
	\open
		\hypo{p}{P}\by{(want $R$)}{}
		\have{q}{Q}\ce{pq,p}
		\have{r}{R}\ce{qr,q}
	\close
	\have{pr}{P \eif R}\ci{p-r}
\end{proof}



\subsection{Biconditional}
Biconditionals indicate that the two sides have the same truth value. One establishes a biconditional by establishing each direction of it as conditionals. To derive $W \eiff X$, for instance, you must establish both $W \eif X$ \emph{and} $X \eif W$. Those conditionals may occur in either order; they need not be on consecutive lines. (Compare the shape of the {\eand}I rule.) Schematically, the rule works like this:


\begin{proof}
	\have[m]{ab}{\metaA{}\eif\metaB{}}
	\have[n]{ba}{\metaB{}\eif\metaA{}}
	\have[\ ]{c}{\metaA{}\eiff\metaB{}} \bi{ab, ba}
\end{proof}



The biconditional elimination rule ({\eiff}E) is a generalized version of \emph{modus ponens} ({\eif}E). If you have the left-hand subsentence of the biconditional, you can derive the right-hand subsentence. If you have the right-hand subsentence, you can derive the left-hand subsentence. This is the rule:



\begin{multicols}{2}
\begin{proof}
	\have[m]{ab}{\metaA{}\eiff\metaB{}}
	\have[n]{a}\metaA{}
	\have[\ ]{b}\metaB{} \be{ab,a}
\end{proof}
\begin{proof}
	\have[m]{ab}{\metaA{}\eiff\metaB{}}
	\have[n]{a}\metaB{}
	\have[\ ]{b}\metaA{} \be{ab,a}
\end{proof}
\end{multicols}



\subsection{Negation}
Here is a simple mathematical argument in English:
\begin{earg}
\item[] Assume there is some greatest natural number. Call it $A$.
\item[] That number plus one is also a natural number.
\item[] Obviously, $A+1 > A$.
\item[] So there is a natural number greater than $A$.
\item[] This is impossible, since $A$ is assumed to be the greatest natural number.
\item[\therefore] There is no greatest natural number.
\end{earg}
This argument form is traditionally called a \emph{reductio}. Its full Latin name is \emph{reductio ad absurdum}, which means `reduction to absurdity.' In a reductio, we assume something for the sake of argument--- for example, that there is a greatest natural number. Then we show that the assumption leads to two contradictory sentences--- for example, that $A$ is the greatest natural number and that it is not. In this way, we show that the original assumption must have been false.

The basic rules for negation will allow for arguments like this. If we assume something and show that it leads to contradictory sentences, then we have proven the negation of the assumption. This is the negation introduction ({\enot}I) rule:

\begin{proof}
\open
	\hypo[m]{na}\metaA{}\by{(for reductio)}{}
	\have[n]{b}\metaB{}
	\have[o]{nb}{\enot\metaB{}}
\close
\have[p]{a}[\ ]{\enot\metaA{}}\ni{na-b, na-nb}
\end{proof}

The {\enot}I rule discharges the assumption for reductio, concluding its negation, when it's shown that some sentence and its negation each follow from the assumption. It cites two (overlapping) ranges: a subproof from the assumption to some sentence \metaB{}, and a subproof from that same assumption to \enot\metaB{}. We write `for reductio' to the right of the assumption, as a note to ourselves, a reminder of why we started the subproof. It is not formally part of the proof, but it is helpful for thinking clearly about the proof.

To see how the rule works, suppose we want to prove an instance of the law of non-contradiction: $\enot(G \eand \enot G)$. We can prove this without any premises by immediately starting a subproof. We want to apply {\enot}I to the subproof, so we assume $(G \eand \enot G)$. We then get an explicit contradiction by {\eand}E. The proof looks like this:

\begin{proof}
	\open
		\hypo{gng}{G\eand \enot G}\by{for reductio}{}
		\have{g}{G}\ae{gng}
		\have{ng}{\enot G}\ae{gng}
	\close
	\have{ngng}{\enot(G \eand \enot G)}\ni{gng-g, gng-ng}
\end{proof}

The {\enot}E rule will work in much the same way. If we assume \enot\metaA{} and show that it leads to a sentence and its negation, we have effectively proven \metaA{}. So the rule looks like this:

\begin{proof}
\open
	\hypo[m]{na}{\enot\metaA{}}\by{for reductio}{}
	\have[n]{b}\metaB{}
	\have[o]{nb}{\enot\metaB{}}
\close
\have[p]{a}[\ ]\metaA{}\ne{na-b, na-nb}
\end{proof}


\section{Derived rules}
The rules of the natural deduction system are meant to be systematic. There is an introduction and an elimination rule for each logical operator, but why these basic rules rather than some others? Some natural deduction systems have a disjunction elimination rule that works like this:

\begin{proof}
	\have[m]{ab}{\metaA{}\eor\metaB{}}
	\have[n]{ac}{\metaA{}\eif\metaC{}}
	\have[o]{bc}{\metaB{}\eif\metaC{}}
	\have[\ ]{c}{\metaC{}} \by{DIL}{ab,ac,bc}
\end{proof}

Let's call this rule Dilemma (DIL). It might seem as if there will be some proofs that we cannot do with our proof system, because we do not have this as a basic rule. Yet this is not the case. Any proof that you can do using the Dilemma rule can be done with basic rules of our natural deduction system. Consider a proof of this form:

\begin{proof}
	\hypo{ab}{\metaA{}\eor\metaB{}}
	\hypo{ac}{\metaA{}\eif\metaC{}}
	\hypo{bc}{\metaB{}\eif\metaC{}}\by{want \metaC{}}{}
	\open
		\hypo{nc}{\enot \metaC{}}\by{for reductio}{}
		\open
			\hypo{a1}\metaA{}\by{for reductio}{}
			\have{c1}{\metaC{}}\ce{ac, a1}
			\have{nc1}{\enot\metaC{}}\by{R}{nc}
		\close
		\have{na}{\enot\metaA{}}\ni{a1-c1, a1-nc1}
		\open
			\hypo{b2}\metaB{}\by{for reductio}{}
			\have{c2}{\metaC{}}\ce{bc, b2}
			\have{nc2}{\enot\metaC{}}\by{R}{nc}
		\close
		\have{b}\metaB{}\oe{ab, na}
		\have{nb}{\enot\metaB{}}\ni{b2-c2, b2-nc2}
	\close
	\have{c}{\metaC{}} \ne{nc-b, nc-nb}
\end{proof}

Remember once again that \metaA{}, \metaB{}, and \metaC{} are meta-variables. They are not symbols of SL, but stand-ins for arbitrary sentences of SL. So this is not, strictly speaking, a proof in SL. It is more like a recipe. It provides a pattern that can prove anything that the Dilemma rule can prove, using only the basic rules of SL. This means that the Dilemma rule is not really necessary. Adding it to the list of basic rules would not allow us to derive anything that we could not derive without it.

Nevertheless, the Dilemma rule would be convenient. It would allow us to do in one line what requires eleven lines and several nested subproofs with the basic rules. So we will add it to the proof system as a derived rule.

A \define{derived rule} is a rule of proof that does not make any new proofs possible. Anything that can be proven with a derived rule can be proven without it. You can think of a short proof using a derived rule as shorthand for a longer proof that uses only the basic rules. Anytime you use the Dilemma rule, you could always take ten extra lines and prove the same thing without it.

For the sake of convenience, we will add several other derived rules. One is \emph{modus tollens} (MT).

\begin{proof}
	\have[m]{ab}{\metaA{}\eif\metaB{}}
	\have[n]{a}{\enot\metaB{}}
	\have[\ ]{b}{\enot\metaA{}} \by{MT}{ab,a}
\end{proof}

We leave the proof of this rule as an exercise. Note that if we had already proven the MT rule, then the proof of the DIL rule could have been done in only five lines.

We also add hypothetical syllogism (HS) as a derived rule. We have already given a proof of it on p.~\pageref{HSproof}.

\begin{proof}
	\have[m]{ab}{\metaA{}\eif\metaB{}}
	\have[n]{bc}{\metaB{}\eif\metaC{}}
	\have[\ ]{ac}{\metaA{}\eif\metaC{}}\by{HS}{ab,bc}
\end{proof}


\section{Rules of replacement}

Consider how you would prove this argument valid: $F\eif(G\eand H)$ \therefore\ $F\eif G$

Perhaps it is tempting to write down the premise and apply the {\eand}E rule to the conjunction $(G \eand H)$. This is impermissible, however, because the basic rules of proof can only be applied to whole sentences. In order to use {\eand}E, we need to get the conjunction $(G \eand H)$ on a line by itself. Here is a proof:

\begin{proof}
	\hypo{fgh}{F\eif(G\eand H)}
	\open
		\hypo{f}{F}\by{want $G$}{}
		\have{gh}{G \eand H}\ce{fgh,f}
		\have{g}{G}\ae{gh}
	\close
	\have{fg}{F \eif G}\ci{f-g}
\end{proof}

The rules we have seen so far must apply to wffs that are on a proof line by themselves. We will now introduce some derived rules that may be applied to part of a sentence. These are called \define{rules of replacement}, because they can be used to replace part of a sentence with a logically equivalent expression. One simple rule of replacement is commutativity (abbreviated Comm), which says that we can swap the order of conjuncts in a conjunction or the order of disjuncts in a disjunction. We define the rule this way:

\begin{center}
\begin{tabular}{rl}
$(\metaA{}\eand\metaB{}) \Longleftrightarrow (\metaB{}\eand\metaA{})$\\
$(\metaA{}\eor\metaB{}) \Longleftrightarrow (\metaB{}\eor\metaA{})$\\
$(\metaA{}\eiff\metaB{}) \Longleftrightarrow (\metaB{}\eiff\metaA{})$
& Comm
\end{tabular}
\end{center}

The double arrow means that you can take a subformula on one side of the arrow and replace it with the subformula on the other side. The arrow is double-headed because rules of replacement work in both directions.

Consider this argument: $(M \eor P) \eif (P \eand M)$, \therefore\ $(P \eor M) \eif (M \eand P)$

It is possible to give a proof of this using only the basic rules, but it will be long and inconvenient. With the Comm rule, we can provide a proof easily:

\begin{proof}
	\hypo{1}{(M \eor P) \eif (P \eand M)}
	\have{2}{(P \eor M) \eif (P \eand M)}\by{Comm}{1}
	\have{n}{(P \eor M) \eif (M \eand P)}\by{Comm}{2}
\end{proof}

Another rule of replacement is double negation (DN). With the DN rule, you can remove or insert a pair of negations for any wff in a line, even if it isn't the whole line. This is the rule:

\begin{center}
\begin{tabular}{rl}
$\enot\enot\metaA{} \Longleftrightarrow \metaA{}$ & DN
\end{tabular}
\end{center}

Two more replacement rules  are called De Morgan's Laws, named for the 19th-century British logician August De Morgan. (Although De Morgan did formalize and publish these laws, many others discussed them before him.) The rules capture useful relations between negation, conjunction, and disjunction. Here are the rules, which we abbreviate DeM:

\begin{center}
\begin{tabular}{rl}
$\enot(\metaA{}\eor\metaB{}) \Longleftrightarrow (\enot\metaA{}\eand\enot\metaB{})$\\
$\enot(\metaA{}\eand\metaB{}) \Longleftrightarrow (\enot\metaA{}\eor\enot\metaB{})$
& DeM
\end{tabular}
\end{center}

As we have seen, $\metaA{}\eif\metaB{}$ is equivalent to $\enot\metaA{}\eor\metaB{}$. A further replacement rule captures this equivalence. We abbreviate the rule MC, for `material conditional.' It takes two forms:

\begin{center}
\begin{tabular}{rl}
$(\metaA{}\eif\metaB{}) \Longleftrightarrow (\enot\metaA{}\eor\metaB{})$ &\\
$(\metaA{}\eor\metaB{}) \Longleftrightarrow (\enot\metaA{}\eif\metaB{})$ & MC
\end{tabular}
\end{center}

Now consider this argument: $\enot(P \eif Q)$, \therefore\ $P \eand \enot Q$

As always, we could prove this argument valid using only the basic rules. With rules of replacement, though, the proof is much simpler:

\begin{proof}
	\hypo{1}{\enot(P \eif Q)}
	\have{2}{\enot(\enot P \eor Q)}\by{MC}{1}
	\have{3}{\enot\enot P \eand \enot Q}\by{DeM}{2}
	\have{4}{P \eand \enot Q}\by{DN}{3}
\end{proof}

A final replacement rule captures the relation between conditionals and biconditionals. We will call this rule biconditional exchange and abbreviate it {\eiff}{ex}.

\begin{center}
\begin{tabular}{rl}
$[(\metaA{}\eif\metaB{})\eand(\metaB{}\eif\metaA{})] \Longleftrightarrow (\metaA{}\eiff\metaB{})$
& {\eiff}{ex}
\end{tabular}
\end{center}


%Although they don't do it in the book, I've been in the habit of writing $(\metaA{}\eand\metaB{}\eand\metaC{})$ and dropping the inner pair of parentheses. This is fine. If we'd wanted to, we could have defined the basic rules in a more general way:

%\begin{proof}
%	\have[n]{a1}{\metaA{}_1}
%	\have{2}{\metaA{}_2}
%	\have[\vdots]{1}{\vdots}
%	\have[n]{an}{\metaA{}_n}
%	\have[\ ]{aaa}{\metaA{}_1~\eand\ldots\eand~\metaA{}_n} \ai{}
%\end{proof}

%\bigskip
%\begin{proof}
%	\have{3}{\metaA{}_1~\eand\ldots\eand~\metaA{}_n}
%	\have{1}{\metaA{}_i} \ae{}
%\end{proof}

%\bigskip
%\begin{proof}
%	\have{1}\metaA{}
%	\have{3}{\metaA{}\eor\metaB{}_1\eor\metaB{}_2\ldots\eor\metaB{}_n} \ai{}
%\end{proof}

%We don't need these extended versions, since for any given n we could prove them as a derived rule.


%The basic rules for conjunction can be valuable in a proof even if there are no conjunctions in any of the assumptions; the basic rules for disjunction can be used even if there are no disjunctions in any assumptions; and similarly for the other basic rules. The rules for identity are different, in that there must be an identity claim in some assumption in order for the rules to do any work. Other than the trivial identity that we can introduce with the {=}I rule


%do not apply we can now prove that identity is \emph{transitive}: If $a=b$ and $b=c$, then $a=c$. The proof proceeds in this way:
%\begin{proof}
%	\open
%		\hypo{p}{a=b \eand b=c}\by{want $a=c$}{}
%		\have{ab}{a=b}\ae{p}
%		\have{bc}{b=c}\ae{p}
%		\have{ac}{a=c}\by{{=}E}{ab,bc}
%	\close
%	\have{conc}{(a=b \eand b=c)\eif a=c} \ci{p-ac}
%\end{proof}


%As an example, consider this argument:
%\begin{quote}
%There is only one button in my pocket. There is a blue button in my pocket. Therefore, there is no button in my pocket that is not blue.
%\end{quote}
%We begin by defining a symbolization key:
%\begin{ekey}
%\item{UD:} buttons in my pocket
%\item{Bx:} $x$ is blue.
%\end{ekey}
%\begin{proof}
%	\hypo{one}{\forall x\forall y\ x=y}
%	\hypo{eb}{\exists x Bx} \by{want $\enot\exists x \enot Bx$}{}
%	\open
%		\hypo{be1}{Be}
%		\have{ef1}{e=f}\Ae{one}
%		\have{bf1}{Bf}\by{{=}E}{ef1,be1}
%	\close
%	\have{bf}{Bf}\Ee{eb,be1-bf1}
%	\have{ab}{\forall x Bx}\Ai{bf}
%	\have{nnab}{\enot\enot\forall x Bx}\by{DN}{ab}
%	\have{nenb}{\enot\exists x\enot Bx}\by{QN}{nnab}
%\end{proof}



\section{Proof strategy}
\label{sec.SL.ND.strategy}

There is no simple recipe for proofs, and there is no substitute for practice. Here, though, are some rules of thumb and strategies to keep in mind.

\paragraph{Work backwards from what you want.}
The ultimate goal is to derive the conclusion. Look at the conclusion and ask what the introduction rule is for its main logical operator. This gives you an idea of what should happen \emph{just before} the last line of the proof. Then you can treat this line as if it were your goal. Ask what you could do to derive this new goal.

For example: If your conclusion is a conditional $\metaA{}\eif\metaB{}$, plan to use the {\eif}I rule. This requires starting a subproof in which you assume \metaA{}. In the subproof, you want to derive \metaB{}.

\paragraph{Work forwards from what you have.}
When you are starting a proof, look at the premises; later, look at the sentences that you have derived so far. Think about the elimination rules for the main operators of these sentences. These will tell you what your options are.

For example: If you have a conditional \metaA{}\eif\metaB{}, and you also have \metaA{}, {\eif}E is a pretty natural choice.

For a short proof, you might be able to eliminate the premises and introduce the conclusion. A long proof is formally just a number of short proofs linked together, so you can fill the gap by alternately working back from the conclusion and forward from the premises.


\paragraph{Change what you are looking at.}
Replacement rules can often make your life easier. If a proof seems impossible, try out some different substitutions.

For example: It is often difficult to prove a disjunction using the basic rules. If you want to show $\metaA{}\eor\metaB{}$, it is often easier to show $\enot\metaA{}\eif\metaB{}$ and use the MC rule.

Some replacement rules should become second nature. If you see a negated disjunction, for instance, you should immediately think of DeMorgan's rule.

\paragraph{Do not forget indirect proof.}
If you cannot find a way to show something directly, try assuming its negation.

Remember that most proofs can be done either indirectly or directly. One way might be easier--- or perhaps one sparks your imagination more than the other--- but either one is formally legitimate.

\paragraph{Repeat as necessary.} Once you have decided how you might be able to get to the conclusion, ask what you might be able to do with the premises. Then consider the target sentences again and ask how you might reach them.

\paragraph{Persist.}
Try different things. If one approach fails, then try something else.




\section{Proof-theoretic concepts}

As we did in our discussion of trees, we will again use the symbol `$\vdash$' to indicate provability. Provability is relative to a proof system, so the meaning of the `$\vdash$' symbol featured in this chapter should be distinguished from the one we used for trees. When necessary, we can specify the single turnstile with reference to the proof system in question, letting `$\vdash_{T}$' stand for provability in the tree system, and `$\vdash_{ND}$' stand for provability in this natural deduction system. For the most part in this chapter, though, we'll be interested in natural deduction, so unless it is specified otherwise, you can understand `$\vdash$' to mean `$\vdash_{ND}$'.

The {double turnstile} symbol `$\models$', remains unchanged. It stands for semantic entailment, as described in ch.~\ref{ch.SLmodels}.

When we write $\{\metaA{}_1,\metaA{}_2,\ldots\}\vdash_{ND}\metaB{}$, this means that it is possible to give a natural deduction proof of \metaB{} with $\metaA{}_1$,$\metaA{}_2$,$\ldots$ as premises. With just one premise, we leave out the curly braces, so $\metaA{}\vdash\metaB{}$ means that there is a proof of \metaB{} with \metaA{} as a premise. Naturally, $\vdash\metaA{}$ means that there is a proof of \metaA{} that has no premises. You can think of it as shorthand for $\emptyset\vdash\metaA{}$. 

For notational completeness, we can understand $\metaSetX{}\vdash\bot$ to mean that from \metaSetX{}, we could prove an arbitrary contradiction. In other words, $\metaSetX{}$ is an inconsistent set.

Logical proofs are sometimes called \emph{derivations}. So $\metaA{}\vdash\metaB{}$ can be read as `\metaB{} is derivable from \metaA{}.'

A \define{theorem} is a sentence that is derivable without any premises; i.e., \metaA{} is a theorem if and only if $\vdash\metaA{}$.

It is not too hard to show that something is a theorem--- you just have to give a proof of it. How could you show that something is \emph{not} a theorem? If its negation is a theorem, then you could provide a proof. For example, it is easy to prove $\enot(P \eand \enot P)$, which shows that $(P \eand \enot P)$ cannot be a theorem. For a sentence that is neither a theorem nor the negation of a theorem, however, there is no easy way to show this. You would have to demonstrate not just that certain proof strategies fail, but that no proof is possible. Even if you fail in trying to prove a sentence in a thousand different ways, perhaps the proof is just too long and complex for you to make out. As we've emphasized already, this is a difference between our natural deduction system and the tree method.

Two sentences \metaA{} and \metaB{} are \define{provably equivalent} if and only if each can be derived from the other; i.e., $\metaA{}\vdash\metaB{}$ and $\metaB{}\vdash\metaA{}$.

It is relatively easy to show that two sentences are provably equivalent--- it just requires a pair of proofs. Showing that sentences are \emph{not} provably equivalent would be much harder. It would be just as hard as showing that a sentence is not a theorem. (In fact, these problems are interchangeable. Can you think of a sentence that would be a theorem if and only if \metaA{} and \metaB{} were provably equivalent?)

The set of sentences $\{\metaA{}_1,\metaA{}_2,\ldots\}$ is \define{provably inconsistent} if and only if contradictory sentences are derivable from it; i.e., for some sentence \metaB{}, $\{\metaA{}_1,\metaA{}_2,\ldots\}\vdash\metaB{}$ and $\{\metaA{}_1,\metaA{}_2,\ldots\}\vdash\enot \metaB{}$. This is equivalent to $\{\metaA{}_1,\metaA{}_2,\ldots\}\vdash\bot$.

It is easy to show that a set is provably inconsistent: You just need to assume the sentences in the set and prove a contradiction. Showing that a set is \emph{not} provably inconsistent will be much harder. It would require more than just providing a proof or two; it would require showing that proofs of a certain kind are \emph{impossible}.


\section{Proofs and models}
As you might already suspect, there is a connection between \emph{theorems} and \emph{tautologies}.

There is a formal way of showing that a sentence is a theorem: Prove it. For each line, we can check to see if that line follows by the cited rule. It may be hard to produce a twenty line proof, but it is not so hard to check each line of the proof and confirm that it is legitimate--- and if each line of the proof individually is legitimate, then the whole proof is legitimate. Showing that a sentence is a tautology, though, requires reasoning in English about all possible models. There is no formal way of checking to see if the reasoning is sound. Given a choice between showing that a sentence is a theorem and showing that it is a tautology, it would be easier to show that it is a theorem.

By contrast, there is no formal way of showing that a sentence is \emph{not} a theorem. We would need to reason in English about all possible proofs. Yet there is a formal method for showing that a sentence is not a tautology. We need only construct a model in which the sentence is false. Given a choice between showing that a sentence is not a theorem and showing that it is not a tautology, it would be easier to show that it is not a tautology.

Fortunately, a sentence is a theorem if and only if it is a tautology. If we provide a proof of $\vdash\metaA{}$ and thus show that it is a theorem, it follows that \metaA{} is a tautology; i.e., $\models\metaA{}$. Similarly, if we construct a model in which \metaA{} is false and thus show that it is not a tautology, it follows that \metaA{} is not a theorem.

In general, $\metaA{}\vdash\metaB{}$ if and only if $\metaA{}\models\metaB{}$. As such:
\begin{itemize}
\item An argument is \emph{valid} if and only if \emph{the conclusion is derivable from the premises}.
\item Two sentences are \emph{logically equivalent} if and only if they are \emph{provably equivalent}.
\item A set of sentences is \emph{consistent} if and only if it is \emph{not provably inconsistent}.
\end{itemize}
You can pick and choose when to think in terms of proofs and when to think in terms of models, doing whichever is easier for a given task. Table \ref{table.ProofOrModel} summarizes when it is best to give proofs and when it is best to give models.

In this way, proofs and models give us a versatile toolkit for working with arguments. If we can translate an argument into SL, then we can measure its logical weight in a purely formal way. If it is deductively valid, we can give a formal proof; if it is invalid, we can provide a formal counterexample.

\begin{table}
\begin{center}
\begin{tabular*}{\textwidth}{p{10em}|p{10em}|p{10em}|}
\cline{2-3}

 & {\centerline{YES}} & {\centerline{NO}}\\
\cline{2-3}

Is \metaA{} a tautology? & prove $\vdash\metaA{}$ & give a model in which \metaA{} is false\\
\cline{2-3}

Is \metaA{} a contradiction? &  prove $\vdash\enot\metaA{}$ & give a model in which \metaA{} is true\\
\cline{2-3}

Is \metaA{} contingent? & give a model in which \metaA{} is true and another in which \metaA{} is false & prove $\vdash\metaA{}$ or $\vdash\enot\metaA{}$\\
\cline{2-3}

Are \metaA{} and \metaB{} equivalent? & prove \mbox{$\metaA{}\vdash\metaB{}$} and \mbox{$\metaB{}\vdash\metaA{}$}  & give a model in which \metaA{} and \metaB{} have different truth values\\
\cline{2-3}

Is the set \model{A} consistent? & give a model in which all the sentences in \model{A} are true & taking the sentences in \model{A}, prove \metaB{} and \enot\metaB{}\\
\cline{2-3}

Is the argument \mbox{`\metaA{}, \metaB{} \ldots \therefore\ \metaC{}'} valid? & prove $\metaA{}, \metaB{}, \ldots \vdash\metaC{}$ & give a model in which \{\metaA{}, \metaB{}, \ldots \} is satisfied and \metaC{} is falsified\\
\cline{2-3}
\end{tabular*}
\end{center}
\caption{Sometimes it is easier to show something by providing proofs than it is by providing models. Sometimes it is the other way round.  It depends on what you are trying to show.}
\label{table.ProofOrModel}
\end{table}

\FloatBarrier

\section{Soundness and completeness}

Chapter \ref{ch.SLsoundcomplete} considered the soundness and completeness of the tree method at length; it proved that this method was both sound ($\metaSetX{}\vdash_{T}\metaA{}$ only if $\metaSetX{}\models{}$\metaA{}) and complete ($\metaSetX{}\models{}$\metaA{} only if $\metaSetX{}\vdash_{T}\metaA{}$). The natural deduction system of this chapter is also both sound and complete for SL. In other words, $\metaSetX{}\vdash_{ND}\metaA{}$ if and only if $\metaSetX{}\models{}\metaA{}.$ Given the soundness and completeness of the tree method, this also means that our two proof systems are equivalent in the sense that anything provable in one is also provable in the other ($\metaSetX{}\vdash_{T}\metaA{})$ iff ($\metaSetX{}\vdash_{ND}\metaA{}$).

How can we know that our natural deduction method is sound? A proof system is \define{sound} if there are no derivations corresponding to invalid arguments. Demonstrating that the proof system is sound would require showing that any possible proof in our system is the proof of a valid argument. There is a fairly simple way of approaching this in a step-wise fashion. If using the {\eand}E rule on the last line of a proof could never change a valid argument into an invalid one, then using the rule many times could not make an argument invalid. Similarly, if using the {\eand}E and {\eor}E rules individually on the last line of a proof could never change a valid argument into an invalid one, then using them in combination could not either.

The strategy is to show for every rule of inference that it alone could not make a valid argument into an invalid one. It follows that the rules used in combination would not make a valid argument invalid. Since a proof is just a series of lines, each justified by a rule of inference, this would show that every provable argument is valid.

Consider, for example, the {\eand}I rule. Suppose we use it to add \metaA{}\eand\metaB{} to a valid argument. In order for the rule to apply, \metaA{} and \metaB{} must already be available in the proof. Since the argument so far is valid, \metaA{} and \metaB{} are either premises of the argument or valid consequences of the premises. As such, any model in which the premises are true must be a model in which \metaA{} and \metaB{} are true. According to the definition of \define{truth in SL}, this means that \metaA{}\eand\metaB{} is also true in such a model. Therefore, \metaA{}\eand\metaB{} validly follows from the premises. This means that using the {\eand}I rule to extend a valid proof produces another valid proof.

In order to show that the proof system is sound, we would need to show this for the other inference rules. Since the derived rules are consequences of the basic rules, it would suffice to provide similar arguments for the 16 other basic rules. The reasoning is extremely similar to that given in the soundness proof for trees in the previous chapter. We will not go through it in detail here.

Given a proof that the proof system is sound, it follows that every theorem is a tautology.

What of completeness? Why think that \emph{every} valid argument is an argument that can be proven in our natural deduction system? That is, why think that $\metaA{}\models\metaB{}$ implies $\metaA{}\vdash\metaB{}$? Our system \emph{is} also complete, but the completeness proof for natural deduction is a bit more complex than the completeness proof for trees. (In the case of trees, we had a mechanical method that was guaranteed to find proofs if they exist; we have seen no such method here, which makes proving these general results harder.) This proof is beyond the scope of this book.

The important point is that, happily, the proof system for SL is both sound and complete. Consequently, we may freely use this natural deduction method to draw conclusions about models in SL.


\section*{Summary of definitions}
\begin{itemize}
\item A sentence \metaA{} is a \define{theorem} if and only if $\vdash\metaA{}$.

\item Two sentences \metaA{} and \metaB{} are \define{provably equivalent} if and only if $\metaA{}\vdash\metaB{}$ and $\metaB{}\vdash\metaA{}$.

\item $\{\metaA{}_1,\metaA{}_2,\ldots\}$ is \define{provably inconsistent} if and only if, for some sentence \metaB{}, $\{\metaA{}_1,\metaA{}_2,\ldots\}\vdash(\metaB{} \eand \enot \metaB{})$.
\end{itemize}



\practiceproblems

\solutions
\problempart
\label{pr.justifySLproof}
Provide a justification (rule and line numbers) for each line of proof that requires one.
\begin{multicols}{2}
\begin{proof}
\hypo{1}{W \eif \enot B}
\hypo{2}{A \eand W}
\hypo{2b}{B \eor (J \eand K)}
\have{3}{W}{}
\have{4}{\enot B} {}
\have{5}{J \eand K} {}
\have{6}{K}{}
\end{proof}

\begin{proof}
\hypo{1}{L \eiff \enot O}
\hypo{2}{L \eor \enot O}
\open
	\hypo{a1}{\enot L}
	\have{a2}{\enot O}{}
	\have{a3}{L}{}
	\have{a4}{\enot L}{}
\close
\have{3}{L}{}
\end{proof}

\begin{proof}
\hypo{1}{Z \eif (C \eand \enot N)}
\hypo{2}{\enot Z \eif (N \eand \enot C)}
\open
	\hypo{a1}{\enot(N \eor  C)}
	\have{a2}{\enot N \eand \enot C} {}
	\open
		\hypo{b1}{Z}
		\have{b2}{C \eand \enot N}{}
		\have{b3}{C}{}
		\have{b4}{\enot C}{}
	\close
	\have{a3}{\enot Z}{}
	\have{a4}{N \eand \enot C}{}
	\have{a5}{N}{}
	\have{a6}{\enot N}{}
\close
\have{3}{N \eor C}{}
\end{proof}
\end{multicols}

\solutions
\problempart
\label{pr.solvedSLproofs}
Give a proof for each argument in SL.
\begin{earg}
\item $K\eand L$, \therefore $K\eiff L$
\item $A\eif (B\eif C)$, \therefore $(A\eand B)\eif C$
\item $P \eand (Q\eor R)$, $P\eif \enot R$, \therefore $Q\eor E$
\item $(C\eand D)\eor E$, \therefore $E\eor D$
\item $\enot F\eif G$, $F\eif H$, \therefore $G\eor H$
\item $(X\eand Y)\eor(X\eand Z)$, $\enot(X\eand D)$, $D\eor M$ \therefore $M$
\end{earg}

\problempart
Give a proof for each argument in SL.
\begin{earg}
\item $Q\eif(Q\eand\enot Q)$, \therefore\ $\enot Q$
\item $J\eif\enot J$, \therefore\ $\enot J$
\item $E\eor F$, $F\eor G$, $\enot F$, \therefore\ $E \eand G$
\item $A\eiff B$, $B\eiff C$, \therefore\ $A\eiff C$
\item $M\eor(N\eif M)$, \therefore\ $\enot M \eif \enot N$
\item $S\eiff T$, \therefore\ $S\eiff (T\eor S)$
\item $(M \eor N) \eand (O \eor P)$, $N \eif P$, $\enot P$, \therefore\ $M\eand O$
\item $(Z\eand K) \eor (K\eand M)$, $K \eif D$, \therefore\ $D$
\end{earg}


\solutions
\problempart
\label{pr.SLND.theorems}
Show that each of the following sentences is a theorem in SL.
\begin{earg}
\item $O \eif O$
\item $N \eor \enot N$
\item $\enot(P\eand \enot P)$
\item $\enot(A \eif \enot C) \eif (A \eif C)$
\item $J \eiff [J\eor (L\eand\enot L)]$
\end{earg}

\problempart
Show that each of the following pairs of sentences are provably equivalent in SL.
\begin{earg}
\item $\enot\enot\enot\enot G$, $G$
\item $T\eif S$, $\enot S \eif \enot T$
\item $R \eiff E$, $E \eiff R$
\item $\enot G \eiff H$, $\enot(G \eiff H)$
\item $U \eif I$, $\enot(U \eand \enot I)$
\end{earg}

\solutions
\problempart
\label{pr.solvedSLproofs2}
Provide proofs to show each of the following.
\begin{earg}
\item $M \eand (\enot N \eif \enot M) \vdash (N \eand M) \eor \enot M$
\item \{$C\eif(E\eand G)$, $\enot C \eif G$\} $\vdash$ $G$
\item \{$(Z\eand K)\eiff(Y\eand M)$, $D\eand(D\eif M)$\} $\vdash$ $Y\eif Z$
\item \{$(W \eor X) \eor (Y \eor Z)$, $X\eif Y$, $\enot Z$\} $\vdash$ $W\eor Y$
\end{earg}



\problempart
For the following, provide proofs using only the basic rules. The proofs will be longer than proofs of the same claims would be using the derived rules.
\begin{earg}
\item Show that MT is a legitimate derived rule. Using only the basic rules, prove the following: \metaA{}\eif\metaB{}, \enot\metaB{}, \therefore\ \enot\metaA{}
\item Show that Comm is a legitimate rule for the biconditional. Using only the basic rules, prove that $\metaA{}\eiff\metaB{}$ and $\metaB{}\eiff\metaA{}$ are equivalent.
\item Using only the basic rules, prove the following instance of DeMorgan's Laws: $(\enot A \eand \enot B)$, \therefore\ $\enot(A \eor B)$
\item Show that {\eiff}{ex} is a legitimate derived rule. Using only the basic rules, prove that $D\eiff E$ and $(D\eif E)\eand(E\eif D)$ are equivalent.
\end{earg}




\problempart
\begin{earg}
\item If you know that $\metaA{}\vdash\metaB{}$, what can you say about $(\metaA{}\eand\metaC{})\vdash\metaB{}$? Explain your answer.
\item If you know that $\metaA{}\vdash\metaB{}$, what can you say about $(\metaA{}\eor\metaC{})\vdash\metaB{}$? Explain your answer.
\end{earg}




